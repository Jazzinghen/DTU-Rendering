\chapter{Week 04}

  \section{Question 01}

    Given a $25W$ light bulb with $20\%$ efficiency we will have that only
    $25W * 0.2 = 5W$ are emitted as light while the rest is emitted as
    heat.

    To compute the number of photons emitted we need to know the energy of
    a single $500nm$ wavelength photon. In order to do that we use the
    following steps:

    \begin{align}
      f &= \frac{c}{\lambda} = \frac{3 \cdot 10^8 \frac{m}{s}}{500 \cdot 10^{-9} m} = 6 \cdot 10^{14} \frac{1}{s} \\
      q[500] &= hf = h \frac{c}{\lambda} = 6.63 \cdot 10^{-34}Js \cdot 6 \cdot 10^{14} \frac{1}{s} \approx 4 \cdot 10^{-19}J \\
      N_{photons}[5W] &= \frac{5W}{4 \cdot 10^{-19}\frac{J}{photon}} = 1.25 \cdot 10^{19} \frac{photons}{s}
    \end{align}

  \section{Question 02}

    To compute the radiant flux $\Phi$ for the given light bulb we use:
    
    \begin{align}
      \Phi &= P = 2.4V \cdot 0.7A = 1.68W
    \end{align}

    To compute radiant intensity $I$ we have to compute the power emitted per unit solid angle:

    \begin{align}\label{eqn:radInt}
      I &= \frac{d\Phi}{d\omega}\frac{W}{sr} \\
        &= \frac{\Phi}{4\pi}\frac{W}{sr}  \\
        &= \frac{1.68}{4\pi}\frac{W}{sr}  \\
        &= 0.1337\frac{W}{sr}
    \end{align}

    To compute radiant Exitance $M$ we have to compute the power emitted per surface unit:

    \begin{align}
      M &= \frac{d\Phi}{dA}\frac{W}{m^2}    \\
        &= \frac{\Phi}{4\pi{}r}\frac{W}{m^2}\\
        &= \frac{1.68}{0.1257}\frac{W}{m^2} \\
        &= 13.369\frac{W}{m^2}
    \end{align}

    To compute the energy emitted in five minutes we just have to multiply the radiant flux (Which is in $W = \frac{J}{s}$)
    for the time:

    \begin{align}
      E &= \Phi{}W \cdot \Delta{}t s \\
        &= 1.68W \cdot 5min \\
        &= 1.68W \cdot 300s \\
        &= 504J
    \end{align}

  \section{Question 03}

    To compute the Irradiance $E$ received by the pupil we have to first compute the solid angle of the pupil in
    respect to the centre of the light bulb then, using that value, get the Energy received by the surface of the pupil
    and then, finally, compute the Irradiance based on the Area of the pupil.

    To compute the solid angle of the pupil $\omega$ we use the following equation:

    \begin{align}\label{eqn:omega}
      \omega  &= \frac{A}{r^2}sr \\
              &= \frac{0.06^2 \cdot \pi m^2}{1 m^2} \\
              &= 0.0113 sr
    \end{align}

    Then, we can use the value computed in equation \ref{eqn:radInt} to compute the Energy $\Phi$:

    \begin{align}\label{eqn:Phi}
      \Phi  &= I \cdot \omega W \\
            &= 0.1337 \frac{W}{sr} \cdot 0.0113 sr \\
            &= 0.0015 W
    \end{align}

    Finally we can compute the Irradiance $E$:

    \begin{align}\label{eqn:Ecomp}
      E &= \frac{\Phi}{A}\frac{W}{m^2} \\
        &= \frac{0.0015W}{0.0113m^2} \\
        &= 0.1327\frac{W}{m^2}
    \end{align}

  \section{Question 04}

    To compute the Irradiance Intensity at the table we have to first compute the Radient Intensity $I$ of the light source, as
    we did in \ref{eqn:radInt}:

    \begin{align}
      I &= \frac{d\Phi}{d\omega}\frac{W}{sr} \\
        &= \frac{\Phi}{4\pi} \\
        &= \frac{200W * 0.2}{4\pi} \\
        &= 3.183 \frac{W}{sr}
    \end{align}

    And now, using this and the formulae \ref{eqn:omega} \ref{eqn:Phi} and \ref{eqn:Ecomp}, we can compute the Irradiance at distance
    $r = 2m$:

    \begin{align}\label{eqn:Ered}
      E &= \frac{\Phi}{A}\frac{W}{m^2}  \\
        &= \frac{\frac{I}{\omega}}{A}\frac{W}{m^2}  \\
        &= \frac{\frac{I}{\frac{A}{r^2}}}{A}\frac{W}{m^2}  \\
        &= \frac{I}{r^2}\frac{W}{m^2}  \\
        &= \frac{3.183\frac{W}{sr}}{4m^2} \\
        &= 0.7958\frac{W}{m^2}
    \end{align}

    Since light with wavelength of $\lambda = 650nm$ has an efficiency value of $0.1$ then we can use the ``Radiometric to Photometric'' equation
    to compute the Illuminance at the table:

    \begin{align}
      E_{photo} &= E_{radio} \cdot 685 \cdot V(\lambda) \\
                &= 0.7958 \cdot 685 \cdot 0.1 \\
                &= 54.51\frac{L}{m^2}
    \end{align}

  \section{Question 05}
    
    When we reach the point in wich both sides are equally illuminated it means that we have the same Irradiance $E_s$ and $E_x$ on both sides.
    As we computed in equation \ref{eqn:Ered} we know that $E = \frac{I}{r^2}$. So, to compute $I_x$ we can do the following:

    \begin{align}
      E_s &= E_x \\
      \frac{I_s}{r_{s}^2} &= \frac{I_x}{r_{x}^2}      \\
      \frac{40cd}{0.35^2m^2} &= \frac{I_x}{0.65^2m^2} \\
      \frac{40cd}{0.35^2m^2} \cdot 0.65^2m^2 &= I_x   \\
      I_x &= \frac{40cd \cdot 0.4225 m^2}{0.1225m^2}  \\
      I_x &= 137.959 cd
    \end{align}

  \section{Question 06}

    To compute the radiosity (Radiant Exitance) $M$ we have to first compute the Flux $\Phi$:
    
    \begin{align}
      L &= \frac{d\Phi}{dA^{\perp}d\omega}        \\
      \Phi &= \int_{\Omega}^{}d\Phi   \\
      \Phi &= LA\pi                   \\
      \Phi &= 5000 \frac{W}{m^2 sr} \cdot \pi \cdot 0.01 m^2 \\
      \Phi &= 157.08 W
    \end{align}

    While the Radiant Exitance $M$ is:

    \begin{align}
      M &= \frac{\Phi}{A}\frac{W}{m^2}  \\
      M &= \frac{156.08 W}{0.01 m^2}    \\
      M &= 15608 \frac{W}{m^2}
    \end{align}

  \section{Question 07}

    
