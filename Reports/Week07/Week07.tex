\chapter{Week 07}
  \image{Week07/NoSub.png}{Scene rendered with no pixel subdivision. Time required: $1s$}{0.5}{img:NoSub}
  \image{Week07/2x2Sub.png}{Scene rendered with 2x2 pixel subdivision. Time required: $3.89s$}{0.5}{img:2x2Sub}
  \image{Week07/3x3Sub.png}{Scene rendered with 3x3 pixel subdivision. Time required: $8.63s$}{0.5}{img:3x3Sub}
  \image{Week07/4x4Sub.png}{Scene rendered with 4x4 pixel subdivision. Time required: $15.31s$}{0.5}{img:4x4Sub}
  \image{Week07/5x5Sub.png}{Scene rendered with 5x5 pixel subdivision. Time required: $24.45s$}{0.5}{img:5x5Sub}
  \image{Week07/6x6Sub.png}{Scene rendered with 6x6 pixel subdivision. Time required: $34.93s$}{0.5}{img:6x6Sub}

  \section{Question 01}
    
    The program cathces thekeyboard input and, pasrsing it, if it gets a ``+'' or ``-'' increases or decreases the
    variable \syntax{subdivs}.

  \section{Question 02}

    Each pixel on screen is divided in subpixels. In the jitter what is stored is a random position inside each of these
    subpixels, used then to displace the various rays casted for every pixel on screen. For each pixel we have $s^2$
    subpixels, as we store the quantity of subpixels per edge.

  \section{Question 04}

    \begin{itemize}
      \item The render time multiplied by the subdivision level to the square
      \item The higher is the subdivion level the lower is the aliasing error
      \item The improvement becomes invisible once the subdivion reaches 3x3 subpixels (i.e. \syntax{subdivs = 3})
      \item Any subdivision higher then 2 is not worth the time waited.
    \end{itemize}

  \section{Code for Ray Jittering}
    \lstinputlisting[language=C++]{Week07/RayCaster.cpp}
