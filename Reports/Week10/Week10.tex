\chapter{Week 10}
  \image{Week10/NearestNoSub.png}{Scene rendered with Caustics, no pixel subdivision and nearest filtering. Time
  required: $8.43s$; Texture Scaling: $0.1x$}{0.5}{img:NearNoSub}

 \image{Week10/Nearest2x2.png}{Scene rendered with Caustics, 2x2 pixel subdivision and nearest filtering. Time
  required: $33.39s$; Texture Scaling: $0.1x$}{0.5}{img:Near2x2}

 \image{Week10/BilinearNoSub.png}{Scene rendered with Caustics, no pixel subdivision and bilinear filtering. Time
  required: $8.83s$; Texture Scaling: $0.1x$}{0.5}{img:BiNoSub}

 \image{Week10/Bilinear2x2.png}{Scene rendered with Caustics, 2x2 pixel subdivision and bilinearfiltering. Time
  required: $34.84s$; Texture Scaling: $0.1x$}{0.5}{img:Bi2x2}

 \image{Week10/Bilinear10x.png}{Scene rendered with Caustics, no pixel subdivision and bilinearfiltering. Time
  required: $8.82s$; Texture Scaling: $10.0x$}{0.5}{img:10xNoSub}

 \image{Week10/Bilinear10x2x2.png}{Scene rendered with Caustics, 2x2 pixel subdivision and bilinearfiltering. Time
  required: $35.04s$; Texture Scaling: $10.0x$}{0.5}{img:10x2x2}

  \section{Question 02}
    
    The colour looked up in the texture is used as the diffuse colour for the surface in that point instead of the
    surface's own diffure colour. In this way we can then compute all the other effects which uses the diffuse component
    to get the final colour.

  \section{Question 06}

    As can be seen in Figures \ref{img:Near2x2} and \ref{img:BiNoSub}, the difference is that in nearest with 2x2 pixel
    subdivision the texture is still much aliased due to the fact that the engine looks up to the nearest texel, whitout
    interpolating, so the small change of landing position given by the jittering is not enough to sample a different
    texel of the texture.

    Bilinear filtering is less aliased but has the side effect of being much more blurry.

  \section{Question 07}
    
    Scaling texture coordinates has the effect of scaling the texture itself. If the scaling value is increased then the
    texture is shrunk and the other way around. As can be seen in Figure \ref{img:Bi2x2}, in which the texture is scaled
    using $0.1$ as factor, and \ref{img:10x2x2}, in which the texture is scaled using $10.0$ as factor, the higher is
    the scaling factor, the smaller the texture will be.

  \section{Random coloured pixels}

    Texture lookup has some random problems and generates some random colour pixels in various positions. This is a
    rendering problem the team did not manage to fix as the functions to load and do lookup of the texels seem correct.
    
    This problem happen only in one direction, however.

  \section{Code for Texturing}
    \lstinputlisting[language=C++]{Week10/Texture.cpp}
