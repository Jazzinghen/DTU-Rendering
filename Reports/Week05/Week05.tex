\chapter{Week 05}

  \section {Part 01}

    The result of running the initial project is shown in Figure \ref{img:Nothing}.

    \image{Week05/RadiosityNothing.png}{Cornel Box Without any modification.}{0.5}{img:Nothing}

  \section {Part 02}

    The first step is to divide the scene into patches, as shown in Figures \ref{img:Wireframe1}
    and \ref{img:Wireframe2}.
    
    \image{Week05/RadiosityCornel1.png}{Wireframe Cornel Box 1 subdivided in patches.}{0.5}{img:Wireframe1}
    \image{Week05/RadiosityPatches.png}{Wireframe Cornel Box 2 subdivided in patches.}{0.5}{img:Wireframe2}

  \section {Part 03}

    After discretizing the scene we computed the radiosity for each patch using the analytical from factor, giving
    the result shown in Figure \ref{img:Discretised}. 

    \image{Week05/RadiosityDiscretizedScene.png}{Discretized Radiosity Rendering.}{0.5}{img:Discretised}
 
  \section {Part 04}
    
    The result shown in Figure \ref{img:Discretised}, however, looks quite ``blocky'' due to the nature
    of our discretized algorithm but smoothing the scene by interpolating the colour between vertices gives better
    results, as shown in Figure \ref{img:Smoothed}. The result, however, doesn't have any shadow.

    \image{Week05/RadiosityCornel2AnalyticMethodSmooth.png}{Nodal averaged colour scene.}{0.5}{img:Smoothed}
    

  \section {Part 05}
    
    Replacing the analytic form factor with the Hemicube form factor computation adds Occlusion to the scene,
    as shown in Figure \ref{img:Hemicube}

    \image{Week05/RadiosityHemicube300.png}{Cornel Box 2 Rendered by using the Hemucube Form Factor computation.}
          {0.5}{img:Hemicube}
    
  \section {Part 06}

    \begin{itemize}
      \item Changing the subdivision level of light sources influences the softness of the shadows: the greater is the
            number of patches a light source is subdivided into the softer the shadows will be. An example of this
            is shown in Figures \ref{img:LightHigh} and \ref{img:LightLow}
      \item When the resolution of the hemicube is getting low compared to the resolution of the patch subdivision some
            artifacts appear, as can be observed in \ref{img:ArtifactsFuckingEverywhere}. This is due to a higher number of
            patches skipped between samples while rendering the scene from the point of view of the hemicube. This effect
            is even more evident in Figure \ref{img:BleedingEyes}. To solve this problem we should adapt the resolution of
            the Hemicube to the patch subdivision size to avoid ``skipping'' patches.
            
            We can also observe, in all the Figures with Hemicube subdivision that the shadows are not precise near the edges of the meshes.
            This is due to the discretization of the scene and can be fixed by increasing a lot the patch subdivision. However
            a better solution could be to adapt the subdivision of the scene to fit these border cases, e.g. by adding verteces near the
            boundaries between meshes. Also adapting the subdivision increasing it on places where the light gradient is strong.
      \item The advantages of Radiosity are that global illumination comes naturally with the computations used by this technique and
            once the ligt has been precomputed the viewer can move through the scene at interactive framerate as long as the scene
            doesn't change.

            On the other hand Radiosity is not suited to render reflection and refractions and is much slower than Raytracing,
            which makes it inadequate for dynamic scenes.
    \end{itemize}

    \image{Week05/RadiosityInsanelyHighLightRes.png}{Cornel Box 2 Rendered by using a greater light source subdivision.}
          {0.5}{img:LightHigh}
    \image{Week05/RadiosityLightFuckingLowRes.png}{Cornel Box 2 Rendered by using a lower light source subdivision.}
          {0.5}{img:LightLow}
    \image{Week05/RadiosityHemicube10.png}{Cornel Box 2 Rendered by using a Hemicube size of 10.}
          {0.5}{img:ArtifactsFuckingEverywhere}
    \image{Week05/RadiosityHemicube10_Explaination.png}{Cornel Box 2 Rendered by using a very high Patch resolution and low Hemicube size.}
          {0.5}{img:BleedingEyes}

  \section {The Code}
    
    \lstinputlisting[language=C++]{Week05/ProgRef.cpp}
